\section{Methodology}
\label{sec:methodology}
% Focus on what you add to the existing method. Explain what you will do and why (and how). Do not forget to characterize your research design. There should be an evaluation plan in this section. (For DS students, this normally means using manually labelled or ground truth data.)
The overarching goal of this research is to adapt and validate a deep learning framework for multi-hazard susceptibility mapping in Europe. 

The first objective is to develop supervised learning models using CNNs for spatial hazards like floods and landslides, and Transformers for temporal hazards like droughts. These models will be trained on pre-engineered features and optimized for predictive accuracy through systematic evaluation.

The second objective focuses on applying transfer learning to address data gaps, particularly for earthquakes. Pre-trained models from Japan will be fine-tuned with European data to improve predictions in regions with limited hazard data.

The third objective is to integrate individual hazard models into a unified multi-hazard map using ensemble methods, capturing interactions between hazards. Validation will use the MYRIAD dataset as a ground truth for validation.

Finally, model outputs will be validated using performance metrics such as ROC-AUC and F1 scores. Interpretability techniques like SHAP values will be used to analyze feature importance and identify key susceptibility patterns across Europe, supporting disaster preparedness efforts.