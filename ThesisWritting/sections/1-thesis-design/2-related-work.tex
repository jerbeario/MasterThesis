\section{Related Work}
\label{sec:related_work}
% Your work needs to be grounded and compared to earlier work and the state-of-the-art. Start the section with announcing the research gap and also end with the research gap. Consider using hypotheses. 
Write about your related work here. Make clear to which key papers you will compare your eventual results. This can be done from the perspective of methods used, the task at hand and the addressed domain.

\subsection{Multi-Hazards: Definitions and Context}
% Concepts: Hazard, multi-hazard
Natural disasters have significant negative impacts on economic, human, and environmental systems. They can cause severe short-term economic disruptions and hinder long-term growth, development, and poverty reduction (Benson \& Clay, 2004). Moreover, a recent shift from considering the effects and consequences of hazards individually has led to a holistic framework for identifying and studying multi-hazards. Defined by the UNDRR as "the specific contexts where hazardous events may occur simultaneously, cascadingly or cumulatively over time, and taking into account the potential interrelated effects'(UNDRR), multi-hazard account for 59 \% of wordlwide economic losses attributable to natural hazards from 1900-2023 (Ryan lee, 2023). 

\subsection{Susceptibility Mapping: Concepts and Methods}
% Concepts: Susceptibility maps, disaster preparedness
% elicit
Hazard susceptibility mapping integrates various datasets and methods to create risk assessment maps, aiding in disaster preparedness and land management (Mausmi Gohil et al., 2024). However, considering the interrelated effects of hazards, interest in multi-hazard susceptibility mapping has grown. These maps assess multiple natural hazards and their interactions, providing a comprehensive view of risk compared to single-hazard approaches (Kashif Ullah et al., 2022; Chuanming Ma et al., 2018). This method combines individual hazard susceptibility maps to create an integrated multi-hazard map, offering decision-makers visual information for effective disaster management and land use planning (Chuanming Ma et al., 2018). Advanced techniques like Convolutional Neural Networks have shown superior performance in predicting multiple hazards compared to conventional machine learning algorithms (Kashif Ullah et al., 2022).
% why we need maps
***
The complex effects of multiple hazards, such as earthquakes, strong winds, and floods, warrant urgent attention for establishing new design code provisions (Tathagata Roy \& Matsagar, 2021).
***



\subsection{European Susceptibility Mapping}
% Concepts: European examples of multi-hazard, disaster preparedness in Europe, susceptibility mapping in Europe examples (local)
Multi-hazard susceptibility mapping in Europe is gaining importance due to the increasing vulnerability of society to various risks. Studies have assessed multiple climate-related hazards across Europe, projecting a progressive increase in overall climate hazard, particularly in southwestern regions, driven by heat waves, droughts, and wildfires (Forzieri et al., 2016). Research has also focused on specific infrastructure, such as power plants, evaluating their susceptibility to earthquakes, floods, tornados, and lightning (Schaefer et al., 2020). Methodological approaches for creating multi-risk maps at regional levels have been developed, integrating stakeholder perceptions with classical risk assessment frameworks (Carpignano et al., 2009)

% Europe has a lot of chemicals, polluting entities than can be released in the wild because of disasters (Natech accidents) 

\subsection{The MYRIAD Hazard Event Set}
% Concepts: MYRIAD-HESA, MYRIAD-HES
The MYRIAD Hazard Event Dataset is a global multi-hazard event set database developed using the MYRIAD-Hazard Event Sets Algorithm (MYRIAD-HESA) (Claassen et al., 2023). This open-access method compiles historically-based multi-hazard event sets from 2004 to 2017, incorporating eleven hazards across various classes. The dataset aims to provide insights into multi-hazard event frequencies and hotspots, considering temporal dimensions like time-lags between hazard occurrences. It supports the MYRIAD-EU project's vision of developing a multi-risk, multi-sector, systemic approach to risk management (Ward, 2021).
\subsection{Deep Learning in Susceptibility Mapping}

\subsection{Research Gap and Opportunity}
% High level idea is to help bridge the gap between policy and science by provided easy to understand susceptibility maps for decisions to be based on.